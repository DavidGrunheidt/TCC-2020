
%%%%%%%%%%%%%%%%%%%%%%%%%%%%%%%%%%%%%%%%%%%%%%%%%%%%%%%%%%%%%%%%%%%%
%%% Lista de acronimos                                           %%%
%%%%%%%%%%%%%%%%%%%%%%%%%%%%%%%%%%%%%%%%%%%%%%%%%%%%%%%%%%%%%%%%%%%%
%%% Importante:                                      
%%% - A lista PRECISA SER MANTIDA ORDENADA
%%%%%%%%%%%%%%%%%%%%%%%%%%%%%%%%%%%%%%%%%%%%%%%%%%%%%%%%%%%%%%%%%%%%

\tnewacronym{APE}{Application Programming Interface}
            % \APE e \APEs nunca espandem, mas ele aparece na lista de siglas
\xnewacronym{DHT}{Distributed Hash Table}
            % Se \DHTs for o primeiro uso, Tabe ganha um s no final
\xnewacronym[][longplural={Square Matrices}]{SQ}{Square Matrix}
               % trata o plural usado em \SQs
               % note que como queremos passar o 2o arg opcional, devemos 
               % passar o primeiro (podemos deixar em branco)
\xnewacronym[WTC]{W3C}{World Wide Web Consortium}
            % gera comando \WTC que expande para "World ... (W3C)"

\xnewacronym{HPC}{High-Performance Computing}

\xnewacronym{API}{Application Programming Interface}

\xnewacronym{IPC}{MPPA Interprocess Communication API}

\xnewacronym{ASYNC}{MPPA Asynchronous Communication API}

\xnewacronym{Flops}{Floating-point Operations per Second}

\xnewacronym{RAM}{Random-Access Memory}

\xnewacronym{CPU}{Central Processing Unit}

\xnewacronym{UMA}{Uniform Memory Access}

\xnewacronym{NUMA}{Nonuniform  Memory  Access}

\xnewacronym{CMP}{Chip MultiProcessadores}

\xnewacronym{GPU}{Unidade de Processamento Gráfico}

\xnewacronym{SIMD}{Single  Instruction  Multiple Data}

\xnewacronym{SO}{Sistema Operacional}

\xnewacronym[][longplural={Clusters de Computação}]{CC}{Cluster de Computação}

\xnewacronym[IO]{E/S}{Entrada e Saída}

\xnewacronym{OpenMP}{Open Multi-Processing}

\xnewacronym{MPI}{Message Passing Interface}

\xnewacronym{SPMD}{Single Program, Multiple Data}

\xnewacronym{NoC}{Network-on-Chip}

\tnewacronym{FPGA}{}

\xnewacronym{UM}{Memória Unificada}

\xnewacronym{FN}{Friendly Numbers}

%%% Local Variables:
%%% mode: latex
%%% TeX-master: "main"
%%% End:
