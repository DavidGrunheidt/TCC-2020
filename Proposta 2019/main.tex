%% abtex2-modelo-trabalho-academico.tex, v-1.7.1 laurocesar
%% Copyright 2012-2013 by abnTeX2 group at http://abntex2.googlecode.com/ 
%%
%% This work may be distributed and/or modified under the
%% conditions of the LaTeX Project Public License, either version 1.3
%% of this license or (at your option) any later version.
%% The latest version of this license is in
%%   http://www.latex-project.org/lppl.txt
%% and version 1.3 or later is part of all distributions of LaTeX
%% version 2005/12/01 or later.
%%
%% This work has the LPPL maintenance status `maintained'.
%% 
%% The Current Maintainer of this work is the abnTeX2 team, led
%% by Lauro César Araujo. Further information are available on 
%% http://abntex2.googlecode.com/
%%
%% This work consists of the files abntex2-modelo-trabalho-academico.tex,
%% abntex2-modelo-include-comandos and abntex2-modelo-references.bib
%%

% ------------------------------------------------------------------------
% ------------------------------------------------------------------------
% abnTeX2: Modelo de Trabalho Academico (tese de doutorado, dissertacao de
% mestrado e trabalhos monograficos em geral) em conformidade com 
% ABNT NBR 14724:2011: Informacao e documentacao - Trabalhos academicos -
% Apresentacao
% ------------------------------------------------------------------------
% ------------------------------------------------------------------------

\documentclass[
    % -- opções da classe memoir --
    12pt,               % tamanho da fonte
    openright,          % capítulos começam em pág ímpar (insere página vazia caso preciso)
    twoside,            % para impressão em verso e anverso. Oposto a oneside
    a4paper,            % tamanho do papel. 
    % -- opções da classe abntex2 --
    %chapter=TITLE,     % títulos de capítulos convertidos em letras maiúsculas
    %section=TITLE,     % títulos de seções convertidos em letras maiúsculas
    %subsection=TITLE,  % títulos de subseções convertidos em letras maiúsculas
    %subsubsection=TITLE,% títulos de subsubseções convertidos em letras maiúsculas
    % -- opções do pacote babel --
    english,            % idioma adicional para hifenização
    brazil,             % o último idioma é o principal do documento
    ]{abntex2}


% ---
% PACOTES
% ---

% ---
% Pacotes fundamentais 
% ---
\usepackage{cmap}               % Mapear caracteres especiais no PDF
\usepackage{lmodern}            % Usa a fonte Latin Modern
\usepackage[T1]{fontenc}        % Selecao de codigos de fonte.
\usepackage[utf8]{inputenc}     % Codificacao do documento (conversão automática dos acentos)
\usepackage{lastpage}           % Usado pela Ficha catalográfica
\usepackage{indentfirst}        % Indenta o primeiro parágrafo de cada seção.
\usepackage{color}              % Controle das cores
\usepackage{graphicx}           % Inclusão de gráficos
\usepackage{xspace}
\usepackage{fourier} 
\usepackage{array}
\usepackage{makecell}
\usepackage{graphics}
% ---

% Pacotes extras
\usepackage[table,xcdraw]{xcolor}   % Inclusão de cores em tabelas
\usepackage{todonotes}
\usepackage{multirow}
\usepackage{pdfpages}
\usepackage{pgfgantt}
\usepackage{changepage}

% Acronyms
\usepackage[acronym,nowarn]{glossaries}

\glsdisablehyper

%%%%%%%%%%%%%%%%%%%%%%%%%%%%%%%%%%%%%%%%%%%%%%%%%%%%%%%%%%%%%%%%%%%%
%%% Lista de acronimos                                           %%%
%%%%%%%%%%%%%%%%%%%%%%%%%%%%%%%%%%%%%%%%%%%%%%%%%%%%%%%%%%%%%%%%%%%%
%%% Importante:                                      
%%% - A lista PRECISA SER MANTIDA ORDENADA
%%%%%%%%%%%%%%%%%%%%%%%%%%%%%%%%%%%%%%%%%%%%%%%%%%%%%%%%%%%%%%%%%%%%

\tnewacronym{APE}{Application Programming Interface}
            % \APE e \APEs nunca espandem, mas ele aparece na lista de siglas
\xnewacronym{DHT}{Distributed Hash Table}
            % Se \DHTs for o primeiro uso, Tabe ganha um s no final
\xnewacronym[][longplural={Square Matrices}]{SQ}{Square Matrix}
               % trata o plural usado em \SQs
               % note que como queremos passar o 2o arg opcional, devemos 
               % passar o primeiro (podemos deixar em branco)
\xnewacronym[WTC]{W3C}{World Wide Web Consortium}
            % gera comando \WTC que expande para "World ... (W3C)"

\xnewacronym{HPC}{High-Performance Computing}

\xnewacronym{API}{Application Programming Interface}

\xnewacronym{IPC}{MPPA Interprocess Communication API}

\xnewacronym{ASYNC}{MPPA Asynchronous Communication API}

\xnewacronym{Flops}{Floating-point Operations per Second}

\xnewacronym{RAM}{Random-Access Memory}

\xnewacronym{CPU}{Central Processing Unit}

\xnewacronym{UMA}{Uniform Memory Access}

\xnewacronym{NUMA}{Nonuniform  Memory  Access}

\xnewacronym{CMP}{Chip MultiProcessadores}

\xnewacronym{GPU}{Unidade de Processamento Gráfico}

\xnewacronym{SIMD}{Single  Instruction  Multiple Data}

\xnewacronym{SO}{Sistema Operacional}

%%% Local Variables:
%%% mode: latex
%%% TeX-master: "main"
%%% End:


% \makeglossaries
% ---
% Pacotes adicionais, usados apenas no âmbito do Modelo Canônico do abnteX2
% ---
\usepackage{lipsum}             % para geração de dummy text
% ---

% ---
% Pacotes de citações
% ---
%\usepackage[brazilian,hyperpageref]{backref}    % Paginas com as citações na bibl
\usepackage[alf]{abntex2cite}                    % Citações padrão ABNT

% --- 
% CONFIGURAÇÕES DE PACOTES
% --- 

% ---
% Informações de dados para CAPA e FOLHA DE ROSTO
% ---
\titulo{Comparação das tecnologias de comunicação entre \textit{clusters} no processador MPPA-256 através do \capb}
\autor{David Grunheidt Vilela Ordine}
\local{Florianópolis}
\data{2019}
\orientador{Márcio Bastos Castro}
\coorientador{Pedro Henrique Penna}
\instituicao{%
  Universidade Federal de Santa Catarina
  \par
  Departamento de Informática e Estatística
  \par
  Ciência da Computação}
\tipotrabalho{Trabalho de Conclusão de Curso de Graduação}
% Preambulo deve conter o tipo do trabalho, o objetivo, o nome da instituição e área de concentração 
\preambulo{Proposta de monografia submetida ao Programa de Graduação em Ciência da Computação para a obtenção do Grau de Bacharel.}
% ---


% ---
% Configurações de aparência do PDF final

% alterando o aspecto da cor azul
\definecolor{blue}{RGB}{41,5,195}

% informações do PDF
\makeatletter
\hypersetup{
        %pagebackref=true,
        pdftitle={\@title}, 
        pdfauthor={\@author},
        pdfsubject={\imprimirpreambulo},
        pdfcreator={LaTeX with abnTeX2},
        pdfkeywords={abnt}{latex}{abntex}{abntex2}{trabalho acadêmico}, 
        hidelinks,
        colorlinks=false,           % false: boxed links; true: colored links
        linkcolor=blue,             % color of internal links
        citecolor=blue,             % color of links to bibliography
        filecolor=magenta,          % color of file links
        urlcolor=blue,
        bookmarksdepth=4
}
\makeatother
% --- 

% --- 
% Espaçamentos entre linhas e parágrafos 
% --- 

% O tamanho do parágrafo é dado por:
\setlength{\parindent}{1.3cm}

% Controle do espaçamento entre um parágrafo e outro:
\setlength{\parskip}{0.2cm}  % tente também \onelineskip

% ---
% compila o indice
% ---
\makeindex
% ---

% ----
% Início do documento
% ----
\begin{document}

% Retira espaço extra obsoleto entre as frases.
\frenchspacing 

% ----------------------------------------------------------
% ELEMENTOS PRÉ-TEXTUAIS
% ----------------------------------------------------------
% \pretextual

% ---
% Capa
% ---
\imprimircapa
% ---

% ---
% Folha de rosto
% ---
\imprimirfolhaderosto*
% ---


% ---
% Inserir folha de aprovação
% ---
%
\begin{folhadeaprovacao}

    \begin{center}
        {\ABNTEXchapterfont\bfseries\Large FOLHA DE APROVAÇÃO DE PROPOSTA DO TCC}
        \hspace{.45\textwidth}
    \end{center}

    \renewcommand\theadalign{cb}
    \renewcommand\theadfont{\bfseries}
    \renewcommand\theadgape{\Gape[4pt]}
    \renewcommand\cellgape{\Gape[4pt]}
    %\def\arraystretch{2}%  1 is the default, change whatever you need
    \begin{flushleft}
    \resizebox{\columnwidth}{!}{%
    \begin{tabular}{|l|l|}
    \hline
            \textbf{Acadêmico(s)} &  David Grunheidt Vilela Ordine \hspace{74mm} \\ \hline
            \textbf{Título do Trabalho}  & \makecell[l]{Comparação das tecnologias de comunicação entre clusters no \\ processador MPPA-256 através do \capb}\\ \hline
            \textbf{Curso} & Ciências da Computação/INE/UFSC \\ \hline
            \textbf{Área de Concentração} & Ciências da Computação \\ \hline
    \end{tabular}
    }
    \end{flushleft}
    \begin {flushleft}
    \textbf{Instruções para preenchimento pelo \emph{ORIENTADOR DO TRABALHO}:}
    \begin{itemize}
    \item Para cada critério avaliado, assinale um X na coluna SIM apenas se considerado aprovado. Caso contrário, indique as alterações necessárias na coluna de Observação.
    \end{itemize}
    \end{flushleft}
    \newcolumntype{C}{>{\centering\arraybackslash}p{25em}}
    \newcolumntype{A}{>{\centering\arraybackslash}p{4em}}
    \newcolumntype{O}{>{\centering\arraybackslash}p{11em}}
    \begin{flushleft}
    \resizebox{\columnwidth}{!}{%
    \begin{tabular}{|C|A|A|A|A|O|}
    \hline
    \multirow{2}{*}{\textbf{Critérios}} & \multicolumn{4}{c|}{\textbf{Aprovado}} & \multirow{2}{*}{\textbf{Observação}} \\[1ex]
    \cline{2-5}
        & \textbf{Sim} & \textbf{Parcial} & \textbf{Não} & \textbf{Não se aplica} & \\
        \hline
        \makecell[l]{O trabalho é adequado para um TCC em CCO \\ (relevância / abrangência)?}& & & & & \\
        \hline
        \makecell[l]{O título é adequado?}& & & & & \\
        \hline
        \makecell[l]{O Tema de pesquisa está claramente descrito?}& & & & & \\
        \hline
        \makecell[l]{O problema/hipóteses de pesquisa do trabalho está \\ claramente identificado?}& & & & & \\
        \hline
        \makecell[l]{A relevância da pesquisa é justificada?}& & & & & \\
        \hline
        \makecell[l]{Os objetivos descrevem completa e claramente o \\ que se pretende alcançar neste trabalho?}& & & & & \\
        \hline
        \makecell[l]{É definido o método a ser adotado no trabalho? \\ O método condiz com os objetivos e é adequado \\ para um TCC?}& & & & & \\
        \hline
        \makecell[l]{Foi definido um cronograma coerente com o método\\ definido (indicando todas as atividades) e com as \\ datas das entregas (p.ex. Projeto I, II, Defesa)?}& & & & & \\
        \hline
        \makecell[l]{Foram identificados custos relativos à execução deste \\ trabalho (se houver)? Haverá financiamento para \\ estes custos?}& & & & & \\
        \hline
        \makecell[l]{Foram identificados todos os envolvidos neste \\ trabalho?}& & & & & \\
        \hline
        \makecell[l]{As formas de comunicação foram definidas?}& & & & & \\[1ex]
        \hline
        \makecell[l]{Riscos potenciais que podem causar desvios do plano \\ foram identificados?}& & & & & \\
        \hline
        \makecell[l]{Caso o TCC envolva a produção de um software ou \\ outro tipo de produto e seja desenvolvido também \\ como uma atividade realizada numa empresa ou \\ laboratório, consta na proposta uma declaração\\ (Anexo 3) de ciência e concordância com a entrega\\ do código fonte e/ou documentação produzidos?}& & & & & \\
        \hline
    \end{tabular}
    }
    \newcolumntype{B}{>{\centering\arraybackslash}p{41.72em}}
    \newcolumntype{E}{>{\centering\arraybackslash}p{7em}}
    \newcolumntype{D}{>{\centering\arraybackslash}p{5em}}
    \newcommand\answerbox{
    \fbox{\rule{0.0001ex}{1pt}\rule[0.1ex]{0pt}{0.0001ex}}\hspace{2mm}}
    \begin{tabular}{|B|}
    \hline
    \makecell[l]{Avaliação}\hspace{35mm} \makecell[c]{\answerbox Aprovado \hspace{15mm} \answerbox Não Aprovado} \\ \hline
    \end{tabular}
    \begin{tabular}{|O|O|D|O|}
    \hline
    \makecell {\textbf{Professor Responsável}} & & & \\ \hline
    \end{tabular}

    \end{flushleft}

    %  ******

    % Descomente esta linha com a folha assinada pelo professor.
    %\includepdf{folha_aceitacao_assinada}
  
\end{folhadeaprovacao}
% ---

% ---
% RESUMOS
% ---

% resumo em português
\begin{resumo}

Os processadores \manycore de baixo consumo energético, tais quais, o \mppa, vieram para solucionar o \textit{trade-off} desvantajoso entre gasto energético e ganho em desempenho que ocorria em alguns supercomputadores. Graças a estes, diversos novos conceitos e arquiteturas foram apresentadas. Porém, devido a diversos fatores, torna-se desafiador implementar aplicações eficientes que tomam total proveito destes processadores. No MPPA-256, especificamente, a quantidade de memória limitada por \textit{cluster} e o não compartilhamento de memória entre \textit{clusters} são alguns destes fatores limitantes.

O objetivo deste trabalho é dar sequencia ao projeto de iniciação científica denominado "Otimização do Benchmark \capb para o processador \manycore de Baixo Consumo Energético \mppa", o qual portou o \capb com a biblioteca \async. Além disso, mudanças serão feitas na lógica da implementação da versão antiga, a qual utilizava a biblioteca \ipc, para que fique equivalente a versão nova, podendo assim comparar aspectos somente das \apis de cada versão.

Serão feitos experimentos no processador \mppa utilizando as aplicações de cada versão do \capb, comparando aspectos de comunicação das \apis de cada versão, tais quais, tempos de execução de cada \textit{cluster}, tempos de execução total da aplicação, quantidade de bytes trocados por cada \textit{cluster} e gasto energético total da aplicação.

Como resultado, espera-se poder apontar, com exatidão, qual das duas tecnologias de comunicação usar em determinado contexto, no \mppa, a partir dos resultados colhidos com as comparações. Assim, aspira-se proporcionar dados concretos para que, futuramente, programadores que utilizem o \mppa saibam quais das duas tecnologias utilizar dependendo do contexto em que sua aplicação esteja inserida.

    \vspace{\onelineskip}
        
    \noindent
    \textbf{Palavras-chaves}: \textit{manycores}. MPPA-256.
    \textit{comunicação}. \textit{async}.

\end{resumo}


% ---
% inserir o sumario
% ---
\pdfbookmark[0]{\contentsname}{toc}
\tableofcontents
\cleardoublepage
% ---


% ----------------------------------------------------------
% ELEMENTOS TEXTUAIS
% ----------------------------------------------------------
\textual

% ----------------------------------------------------------
% Introdução
% ----------------------------------------------------------
% \chapter{Projeto}
% \section{Introdução}
% \label{sec:introducao}

\chapter{Introdução}
\label{cap:introducao}

Na década atual, foi crescente a expectativa de que os supercomputadores pudessem alcançar a computação em \textit{exaescale}. Grande parte destes supercomputadores utilizam processadores \textit{multi-core} do estado da arte em seus \textit{clusters}. Porém, a comunidade científica observou uma barreira no caminho ao tentarem seguir sempre a mesma técnica para o ganho de desempenho. O aumento no número de núcleos de processamento por \textit{chip} foi a técnica seguida na ultima década, o que, em determinado momento, não trouxe mais resultados significativos quanto ao \textit{trade-off} entre consumo de energia e ganho em desempenho.

O Departamento de Defesa do Governo dos Estados Unidos (DARPA), uma das organizações mais importantes do pais, também foi um dos órgãos responsáveis por alertar a comunidade científica acerca desta barreia em um de seus relatórios \cite{darpa:exascale}. Neste relatório fica claro que os supercomputadores atuais estão longe de executar a uma potencia aceitável para atingir o \textit{exaescale}, já que, enquanto o supercomputador mais eficiente do mundo utiliza cerca de 15.1 GFlops/W, o indicado como aceitável pela DARPA é de 50 GFlops/W.

Para solucionar este problema, pesquisadores realizaram diversos estudos a fim de apresentar novos tipos de arquiteturas voltadas a eficiência energética, buscando bom balanceamento entre consumo energético e desempenho, ou seja, seguindo os conceitos de \textit{Green Computing}. Logo, o grande interesse da comunidade científica de \hpc acerca deste tema foi um dos responsáveis por alavancar a produção de novos tipos de processadores, como os processadores \textit{manycore} de baixa potência, por exemplo, o \mppa \cite{MPPA-2:2013}, o Adapteva Epiphany \cite{Olofsson2014} e o SW26010, utilizado no supercomputador \textit{Sunway TaihuLight} \cite{sunway:2016}.

Com a finalidade de avaliar o desempenho e consumo energético do \mppa, \textit{Souza et. al} propuseram um benchmark denominado \capb, o qual utiliza uma \api de comunicação síncrona entre processos denominada \ipc \cite{MPPA-2:2013}.
Esta \api proporciona perda de energia ao utilizar o hardware para sincronização, além de possuir baixo nivel de abstração, requerendo conhecimento prévio da arquitetura alvo para implementações paralelas eficientes. Ao realizar a otimização do \textit{benchmark}, \textit{David et. al} o otimizaram através do porte com a \api \async, a qual possui maior nível de abstração e maior potencial em desempenho. Além disso, alterações na lógica de implementação foram realizadas.

Portanto, faz-se necessário atualizar a versão antiga do \textit{benchmark} quanto a lógica de implementação, para que esta se equivalha a nova implementação, criando assim um ambiente propicio para comparação de aspectos puramente das tecnologias de comunicação citadas, utilizando as duas versões do \textit{benchmark} para isso. Após isso, necessita-se comparar ambas as implementações a fim de determinar qual tecnologia de comunicação se comporta de maneira mais eficiente e robusta no \mppa em determinados contextos, para que tenhamos dados de execução de cada \api numa mesma versão de placa do \mppa, tendo assim provas concretas na hora de escolher uma das duas \apis.

\chapter{Objetivos}
\label{cap:objetivos}

O objetivo deste trabalho é obter dados concretos acerca da execução de aplicações de diversos domínios de problemas no \mppa, utilizando as duas \apis já citadas e o \capb, podendo assim, comparar as execuções de cada aplicação em cada cenário especifico possível dentro do processador, obtendo identificadores precisos que, em momentos futuros, possam apontar qual das duas \apis utilizar, dependendo do domínio de problema de uma certa aplicação.

Assim, podem ser definidos como objetivos específicos desse trabalho:
\begin{enumerate}
    \item Investigar a viabilidade do uso do \mppa para a computação de alto desempenho;
    \item Estudar aspectos das \apis de comunicação existentes no \mppa, como a \async e \ipc;
    \item Implementar, em ambas as versões do \capb, novas métricas na obtenção dos dados de execução das aplicações, relacionados a memória \textit{cache} de cada \textit{chip};
    \item Avaliar os custos e benefícios do \mppa em relação ao desempenho e gasto energético, assim como sua utilidade para a Computação Sustentável (\textit{Green Computing});
    \item Comparar as tecnologias \async e \ipc de comunicação entre processos do \mppa, a fim de prover métricas precisas para escolha de uma das duas na hora de implementar uma nova aplicação para o processador.
\end{enumerate}


\begin{flushleft}
      \textbf{Restrições:}
    \begin{itemize}
        \item Acesso somente virtual ao processador \mppa.
        \item Documentação limitada de ambas as tecnologias de comunicação citadas.
    \end{itemize}
        
    \textbf{Premissas:}
    \begin{itemize}
        \item Disponibilidade da versão antiga do \capb;
        \item O orientador estará disponível para reuniões periódicas;
        \item Computador disponível; 
        \item Acesso remoto ao processador \mppa; 
          \item Processador disponível para execução dos experimentos;
        \item Disponibilidade de água, luz e energia;
        \item Acesso à Internet.
    \end{itemize}
        
    \textbf{Marcos:}
    \begin{itemize}
        \item Entrega do resumo em TCC I: 2ª semana de Junho/2020; 
        \item Primeira entrega da monografia em TCC II: 1ª semana de Novembro/2020; 
        \item Defesa da monografia: 2ª semana de Novembro/2020;
        \item Segunda entrega da monografia em TCC II (versão final): 1ª semana de Dezembro/2020.
    \end{itemize}
        
    \textbf{Critérios de aceite:}
    \begin{itemize}
        \item Aprovação da banca avaliadora;
        \item Aprovação do orientador;
        \item Aprovação do professor responsável;
        \item Conformidade com as normas definidas pela instituição;
        \item Prazos cumpridos;
    \end{itemize}
\end{flushleft}
    

\chapter{Método de Pesquisa}
\label{cap:metodo-pesquisa}

O presente projeto será desenvolvido em sequencia aos trabalhos de pesquisa do orientando, orientador e coorientador deste trabalho, David Grunheidt Vilela Ordine, Márcio Bastos Castro e Pedro Henrique Penna, acerca do \mppa. Todas as contribuições desenvolvidas serão implementadas no repositório aberto do \capb, disponibilizado em \url{https://github.com/cart-pucminas/CAPBenchmarks}.
    
O desenvolvimento do trabalho será feito nas linguagem \textit{C} e \textit{Python}, sendo subdividido em seis etapas:
(i) Estudo acerca dos processadores \manycore de baixa potência do estado da arte e as tecnologias de comunicação entre processos implementadas nestes;
(ii) Melhoria da versão antiga para equiparação a nova versão do \capb; 
(iii) Estudo da nova \api disponibilizada pela \textit{Kalray} para coleta de métricas relacionadas a memória \textit{cache};
(iv) Implementação de novas métricas quanto a execução das aplicações em ambas as versões; 
(v) Implementação dos scripts de coleta de todas as métricas já implementadas; 
(vi) Coleta dos dados de execução de cada aplicação.

A segunda e terceira etapa do trabalho consistirão em intensa consulta as documentações providas pela \textit{Kalray} acerca das \apis \async e \ipc. Além disso, uma nova \api para coletar dados acerca da memoria \textit{cache} de cada \textit{core} também será utilizada, sendo assim, serão consultadas, constantemente, 3 documentações. Já a primeira etapa prevê extensa leitura de artigos que refletem o estado da arte acerca do tema tratado, através de diversas plataformas como \textit{IEEE} e \textit{ACM}. Simplificando, todas as etapas consistirão na continuação do projeto de pesquisa de iniciação científica do orientando, a qual portou o \capb com a \api \async.

A plataforma que contém o \mppa está localizada em Grenoble (França) e o acesso
remoto é proporcionado por uma parceria entre o \lapesd e o \lig. O orientando terá acesso ao laboratório \lapesd, localizado na \ufsc, porém, o desenvolvimento do trabalho será feito em seu computador pessoal, o qual terá acesso remoto a plataforma onde localiza-se o \mppa. 

% ----------------------------------------------------------
% PARTE - preparação da pesquisa
% ----------------------------------------------------------
% \part{Preparação da pesquisa}

% ----------------------------------------------------------
% Capitulo com exemplos de comandos inseridos de arquivo externo 
% ----------------------------------------------------------

\include{abntex2-modelo-include-comandos}

% ----------------------------------------------------------
% Parte de revisão de literatura
% ----------------------------------------------------------
% \part{Revisão de Literatura}

% ---
% Capitulo de revisão de literatura

% ---
\chapter{Cronograma}
\label{cap:cronograma}

    As atividades previstas neste projeto estão descritas abaixo:

    \begin{enumerate}
        \item \textbf{A1: Estudo de fundamentação teórica.} Revisão bibliográfica dos materiais que serão utilizados na pesquisa;
        \item \textbf{A2: Estudo e alteração da versão antiga do \capb.} Estudo da versão antiga do \capb, assim como a \api que esta porta, a \ipc, para que se consiga executar e alterar esta versão de modo que equivalha-se a nova, em termos de lógica do domínio do problema.
        \item \textbf{A3: Estudo e implementação das novas métricas em ambas versões do \capb.} Estudo da nova \api para coleta de dados acerca da memoria \textit{cache} dos \textit{cores} através da inclusão de funções que as constituem em ambas as versões, observando os resultados intermediários e adaptando as implementações conforme o ganho de conhecimento.
        \item \textbf{A4 : Escrita do relatório do TCC I.} Elaboração do relatório de TCC I contendo todos os desenvolvimentos realizados até o momento. Entrega prevista para a segunda semana do mês de Junho;
        \item \textbf{A5: Implementação dos \textit{scripts} para os experimentos.} Estudos acerca dos melhores métodos para coleta dos dados disponiveis em todas as aplicações, assim como das bibliotecas em \textit{Python} que facilitarão a coleta e transformação dos dados.
        \item \textbf{A6: Coleta das métricas implementadas.} Coleta e medição de todos os dados acerca das execuções das aplicações, após a implementação dos \textit{scripts} em A6, e avaliação dos resultados obtidos;
        \item \textbf{A7: Escrita do rascunho de TCC II.} Elaboração do rascunho de TCC II para ser defendido na banca após a coleta dos dados em A7. Entrega prevista para a primeira semana de Novembro;
        \item \textbf{A8: Preparação da defesa pública.} Preparação da apresentação oral do trabalho, bem como do material de apoio a ser utilizado durante a apresentação;
        \item \textbf{A9: Defesa pública.} Realização da defesa do trabalho desenvolvido. Pretende-se realizar a defesa na segunda semana de Novembro.
        \item \textbf{A10: Correções e entrega da versão final do TCC.} Correção e ajustes na monografia com base na avaliação da banca para que se possa fazer a entrega final do documento. Entrega prevista para a primeira semana de Dezembro.
    \end{enumerate}

    A figura \ref{fig:cronograma} apresenta a distribuição esperada ao longo do tempo das atividades 
    descritas anteriormente, que serão realizadas durante o primeiro e o segundo semestres de 2020.

    \vspace{-1.5ex}

    % \begin{adjustwidth}{-2.5cm}{}
    \begin{figure}[h]
        \begin{center}
            \begin{ganttchart}[
                y unit title=0.4cm,
                y unit chart=0.6cm,
                hgrid,
                vgrid={{dotted, dotted, dotted, dotted, dotted, black}},
                title label font=\scriptsize,
                title/.append style={fill=gray!30},
                title height=1,
                bar/.append style={fill=gray!30,rounded corners=2pt},
                bar label font=\scriptsize,
                group label font=\scriptsize,
                ]{1}{26}
                \gantttitle{\textbf{2019}}{2}
                \gantttitle{\textbf{2020}}{24}\\
                \gantttitle{\textbf{Dez}}{2}
                \gantttitle{\textbf{Jan}}{2}
                \gantttitle{\textbf{Fev}}{2}
                \gantttitle{\textbf{Mar}}{2}
                \gantttitle{\textbf{Abr}}{2}
                \gantttitle{\textbf{Mai}}{2}
                \gantttitle{\textbf{Jun}}{2}
                \gantttitle{\textbf{Jul}}{2}
                \gantttitle{\textbf{Ago}}{2}
                \gantttitle{\textbf{Set}}{2}
                \gantttitle{\textbf{Out}}{2}
                \gantttitle{\textbf{Nov}}{2}
                \gantttitle{\textbf{Dez}}{2}\\

                \ganttbar{A1}{1}{6}    \\
                \ganttbar{A2}{1}{8}    \\
                \ganttbar{A3}{1}{10}    \\
                \ganttbar{A4}{1}{14}   \\
                \ganttbar{A5}{11}{15}  \\
                \ganttbar{A6}{13}{15}  \\
                \ganttbar{A7}{15}{21}  \\
                \ganttbar{A8}{22}{23}  \\
                \ganttbar{A9}{24}{24}  \\
                \ganttbar{A10}{25}{25} 
            \end{ganttchart}
            %  \end{adjustwidth}
            \caption{Cronograma de atividades.}
            \label{fig:cronograma}
        \end{center}
    \end{figure}
    % ---

% ---

% ---
% Finaliza a parte no bookmark do PDF, para que se inicie o bookmark na raiz
% ---
\bookmarksetup{startatroot}% 
% ---

% ----------------------------------------------------------
% ELEMENTOS PÓS-TEXTUAIS
% ----------------------------------------------------------
\postextual

% ----------------------------------------------------------
% Referências bibliográficas
% ----------------------------------------------------------
\bibliography{bibliography}

% ----------------------------------------------------------
% Glossário
% ----------------------------------------------------------
%
% Consulte o manual da classe abntex2 para orientações sobre o glossário.
%
%\glossary

%---------------------------------------------------------------------
% INDICE REMISSIVO
%---------------------------------------------------------------------

%\printindex

\end{document}
