\newcommand{\mppa}{MPPA-256\xspace}
\newcommand{\capb}{CAP Bench\xspace}
\newcommand{\epiphany}{Adapteva Epiphany\xspace}
\newcommand{\manycore}{\textit{manycore}\xspace}
\newcommand{\manycores}{\textit{manycores}\xspace}
\newcommand{\bench}{\textit{benchmark}\xspace}

% Assume-se que \pretextual já foi feito

\imprimircapa%
\imprimirfolhaderosto*
% Atenção! esse \protect é importante
\protect\incluirfichacatalografica{ficha.pdf}
\imprimirfolhadecertificacao


\begin{dedicatoria}
  Este trabalho é dedicado a minha família, que sempre me apoiou e esteve do meu lado, e também aos meus amigos, os quais me ajudaram a passar por todo o processo de escrita e implementação de uma maneira mais feliz.
\end{dedicatoria}


\begin{agradecimentos}
  Agradeço a todos os meus colegas de trabalho e de curso, os quais contribuíram significativamente para a conclusão deste trabalho, através da troca de experiência e conhecimentos técnicos. Em especial, agradeço ao meu orientador, Márcio Bastos Castro, e meu co-orientador, Pedro Henrique Penna, por despertarem em mim interesse na área da computação paralela e me ajudarem no processo de aprendizado e desenvolvimento deste trabalho.
\end{agradecimentos}


\begin{epigrafe}
  For a number of years I have been familiar with the observation that the quality of programmers is a decreasing function of the density of go to statements in the programs they produce \\
  \cite{dijkstra1968}
\end{epigrafe}


\begin{resumo}[Resumo]
  O principal método para o ganho em desempenho, no processo de evolução dos processadores \textit{single-core}, foi o aumento da frequência de \textit{clock} do processador, o qual, com a crescente desproporção entre o gasto energético e o aumento de performance, deixou de ser viável. Diz-se então que esta desproporção foi a barreira de evolução para esta classe de processadores. Soluções que utilizam processadores \textit{multi-core}, por exemplo, supercomputadores, também enfrentam uma barreira similar, nos dias de hoje, ao agrupar diversos destes processadores em \textit{clusters}, ou agrupar diversos núcleos em um mesmo \textit{chip}. Processadores \manycore de baixo consumo energético, como o \mppa e o \epiphany, surgiram como uma possível solução para este problema. Entretanto, devido a questões arquiteturais, como uma memória distribuída e limitada no \textit{chip}, a implementação de uma aplicação que beneficia-se totalmente do \textit{hardware} de um processador desta classe mostra-se desafiadora. Porém, quando bem feita, sobressai alguns processadores \textit{multi-core} do estado da arte, através do menor consumo energético. Neste projeto foram propostas para o \capb, um \bench desenvolvido para avaliar o desempenho e o consumo de energia do \mppa, otimizações nas aplicações da versão atual e a criação de uma versão das aplicações que utiliza uma nova tecnologia de comunicação assíncrona entre \textit{clusters}, com objetivo de analisar as duas tecnologias de cada versão. Os resultados até o momento mostram que as aplicações que utilizam a nova biblioteca apresentam melhor desempenho sobre as aplicações da versão antiga. Isso se deve principalmente pela característica assíncrona desta biblioteca.



  % Atenção! a BU exige separação através de ponto (.). Ela recomanda de 3 a 5 keywords
  \vspace{\baselineskip} 
  \textbf{Palavras-chave:} Benchmark. Manycore. Desempenho. Green-Computing.
\end{resumo}


\begin{resumo}[Resumo Estendido]
%%%%%%%%%%%%%%%%%%%%%%%%%%%%%%%%%%%%%%%%%%%%%%%%%%%%%%%%%%%%%%%%%%%%%%
% Atenção: normas e templates contraditórios!!!                    %%%
%%%%%%%%%%%%%%%%%%%%%%%%%%%%%%%%%%%%%%%%%%%%%%%%%%%%%%%%%%%%%%%%%%%%%%
% - Modelo da BU: https://repositorio.ufsc.br/handle/123456789/197458
% - A BU exige no **mínimo** 2 páginas e no **máximo** 5
% - Regimento do PPGCC, Art 40 Entende-se  por  resumo  estendido  um  documento  que  contenha  as  informações  mais  relevantes  de  cada  capítulo  da  tese  ou  da  dissertação.
% O mais seguro é ignorar o regimento e seguir a BU.
    % Atenção! A BU diz que o resumo **deve** conter as seções abaixo!
  \section*{Introdução} % Deve ser  subsection*, devido a formatação usada no modelo

  \section*{Fundamentação Teórica} 

  \section*{Trabalhos Correlatos} 
  
  \section*{Desenvolvimento} 

  \section*{Resultados preliminares} 

  \section*{Conclusão}

  \vspace{\baselineskip}  % Atenção! manter igual ao resumo
  \textbf{Palavras-chave:} Benchmark. Manycore. Desempenho. Green-Computing.
\end{resumo}


\begin{abstract}
Throughout the evolving process of \textit{single-core} processors, the main method to gain performance was to increase the processor \textit{clock} frequency, which led to the growing disproportion between energy consumption and increase in performance, making this method not viable anymore. This disproportion was then the barrier to the evolution of this class of processors. \textit{Multi-core} processors solutions, for example, supercomputers, also face a similar barrier nowadays when grouping this processors into \textit{clusters}, or grouping several \textit{cores} into a single \textit{chip}. Low consumption \manycore processors, for instance, the \mppa and the \epiphany, are arising to solve this problem. However, due to architectural characteristics, such as a limited and distributed memory, implementing applications that fully benefits from the hardware of a processor of this class is not an easy task. Yet, when a good implementation is done, it can outstand state-of-the-art processors, through lower energy consumption. This project proposes to the \capb, a \bench developed to evaluate both \mppa performance and energy consumption, optimizations to its applications and the implementation of a new version, using a new communication technology, based on asynchronous primitives, aiming to analyze the technologies used in each version. The results until now show that the applications that use the new technology have a better performance than the old ones. This is due, mainly, by the asynchronous characteristic of this library.

  \vspace{\baselineskip} 
  \textbf{Keywords:} Benchmark. Manycore. Performance. Green-Computing.
\end{abstract}

\listoffigures*  % O * evita que apareça no sumário
\listoftables*
\listoflistings*  
\listofalgorithms*

\listadesiglas*[5em]

\begin{listadesimbolos}
  $\gets$   & Atribuição \\
  $\exists$   & Quantificação existencial \\
  $\rightarrow$   & Implicação \\
  $\wedge$   & E lógico \\
  $\vee$   & Ou lógico \\
  $\neg$   & Negação lógica \\
  $\mapsto$   & Mapeia para \\
  $\sqsubseteq$   & Subclasse (em ontologias) \\
  $\subseteq$   & Subconjunto: $\forall x\;.\; x \in A \rightarrow x \in B$ \\
  $\langle\ldots\rangle$ & Tupla \\
  $\forall$   & Quantificação universal \\
  mmmmm & Nenhum sentido, apenas estou aqui para demonstrar a largura máxima dessas colunas. Ao abrir o ambiente \texttt{listadesimbolos}, pode-se fornecer um argumento opcional indicando a largura da coluna da esquerda (o default é de 5em): \texttt{\textbackslash{}begin\{listadesimbolos\}[2cm] .... \textbackslash{}end\{listadesimbolos\}} \\
\end{listadesimbolos}

\tableofcontents*%

%%% Local Variables:
%%% mode: latex
%%% TeX-master: "main"
%%% End:
